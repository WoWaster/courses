% !TEX TS-program = lualatex

\documentclass[aspectratio=169, russian]{beamer}

%%% Setup fonts.
\usepackage{fontspec}
%% Fancy fonts
\setmainfont{Iosevka Etoile}
\setsansfont{Iosevka Aile}
\setmonofont{Iosevka}

%% Default fonts
% \setmainfont{CMU Serif}
% \setsansfont{CMU Sans Serif}
% \setmonofont{CMU Typewriter Text}

%% Math
\usepackage{amsmath, amsfonts, amsthm, mathtools} % Advanced math tools.
\usepackage{amssymb}

\usepackage{unicode-math} % Allow TTF and OTF fonts in math and allow direct typing unicode math characters.
\unimathsetup{
    warnings-off={
            mathtools-colon,
            mathtools-overbracket
        }
}
% \setmathfont{Lete Sans Math}[CharacterVariant={3,6},StylisticSet={4}]
\setmathfont{Latin Modern Math}

%%% Language settings.
\usepackage{polyglossia}
\setdefaultlanguage{russian}
\setotherlanguage{english}

%%% Beamer settings
% Themes
\usetheme{Boadilla}
\useinnertheme{circles}
\usecolortheme[style=Latte, accent=Blue]{catppuccin}
% Templates
\setbeamertemplate{navigation symbols}{} %remove navigation symbols
\setbeamertemplate{page number in head/foot}[appendixframenumber]
\setbeamertemplate{title page}[default][colsep=-4bp,rounded=true]

%%% Colors
\hypersetup{colorlinks}
\usepackage{catppuccinpalette}

\usepackage{booktabs}

\usepackage{minted}
\usemintedstyle{catppuccin-latte}

\usepackage{csquotes}

% \usepackage{xurl}

%% Custom commands
\NewDocumentCommand{\attribution}{mmmmm}{\href{#1}{#2}, \href{#3}{#4}, #5}
\NewDocumentCommand{\attributionCCThreeWikimedia}{mm}{\attribution{#1}{#2}{https://creativecommons.org/licenses/by-sa/3.0}{CC BY-SA 3.0}{via Wikimedia Commons}}


\usepackage{xurl}
\usepackage{outlines}

%%% Meta

\title{Файлы и память}
\author{Николай Пономарев}
\date{14 февраля 2026 г.}
\titlegraphic{\includegraphics[height=1cm]{../фирменный блок_серый.pdf}}
\subject{Структуры, указатели, модули и файлы. Общие правила разбиения на модули, модули в C. }

\begin{document}

\begin{frame}[plain, noframenumbering]
    \titlepage
\end{frame}

\section{Введение}

\begin{frame}
    \frametitle{Формальные вопросы}
    \begin{outline}
        \1 Занятия по субботам на второй и третьей паре в 3248
        \2 В расписании так и останется 3389
        \1 Курс на HwProj: \url{https://hwproj.ru/courses/50069}
        \1 Будут три контрольные, на 4-й паре
        \1 Будут \enquote{Летучки}
        \1 В конце курса будут доклады
        \1 \emph{Конец курса в начале апреля!}
        \2 Зачёт будет прямо посреди семестра!
        \1 Каждую неделю будут выбираться несколько \emph{счастливчиков}, которые будут сдавать домашние работы вживую
        \2 Подозрительный на творчество БЯМ код увеличивает шансы быть избранным
    \end{outline}
\end{frame}

\begin{frame}
    \frametitle{Критерии оценивания}
    \begin{outline}
        \1 Баллы:
        \2 За домашки (их будет мало, но объёмные)
        \2 За контрольные --- выбирается две лучшие попытки из трёх
        \3 Плюс переписывания на зачёте, пересдаче, комиссии
        \1 Итоговый балл за домашки: $MAX(0, (\frac{n}{N}\ –\ 0.6)) * 2.5 * 100$
        \2 Примерно 50 баллов максимум за обязательные домашние работы
        \1 Летучки дают дополнительные баллы к домашке (максимум где-то три балла за каждую), которые не считаются в максимум
        \1 Доклады также дают дополнительные баллы (примерно 5), но они достанутся только наиболее обречённым
        \1 Есть дедлайны (минус балл к максимуму за каждую неделю просрочки, но не больше половины баллов)
        \1 В качестве итогового берётся \textbf{минимум} из баллов за домашние работы и контрольные
    \end{outline}
\end{frame}

\begin{frame}
    \frametitle{Примерные баллы}

    \begin{center}
        \begin{tabular}{ccc}
            \toprule
            Балл за домашку & Балл за контрольные & Оценка ECTS  \\ \midrule
            48-50           & 18-20               & A            \\ \midrule
            46-47           & 16-17               & B            \\ \midrule
            44-45           & 14-15               & C            \\ \midrule
            43-44           & 12-13               & D            \\ \midrule
            40-42           & 10-11               & E            \\ \midrule
            0-39            & 0-9                 & на пересдачу \\
            \bottomrule
        \end{tabular}
    \end{center}
\end{frame}

\begin{frame}
    \frametitle{Что будет в курсе}
    \begin{outline}
        \1 Алгоритмы и структуры данных
        \2 Деревья, деревья поиска, самобалансирующиеся деревья
        \2 Графы
        \2 Формальные языки, автоматы и лексический анализ
        \2 Немного больше про сортировки
        \1 Системное программирование
        \2 Немного про то, что, судя по зачёту, не очень зашло на \enquote{Информатике}: файлы, память, функции ОС и как этим пользоваться в программах
        \2 Профиляторы, Perf
    \end{outline}
    Сначала будет просто, потом тяжело, потом снова просто
\end{frame}

\section{Летучка}

\begin{frame}

    \begin{center}
        \LARGE

        \structure{Внезапная летучка}

        \vspace{1em}

        \Large
        Доставайте листочки и ручки
    \end{center}

\end{frame}

\begin{frame}{Внезапная летучка}
    \begin{enumerate}
        \item Напишите команду для того, чтобы получить исполняемый файл \enquote{executable} из программы в файле \enquote{program.c}
        \item Что должно и чего не должно быть в файле с расширением .h? Зачем они нужны?
        \item Напишите функцию, которая принимает массив чисел и его размер и возвращает новый массив, состоящий из квадратов чисел исходного массива
    \end{enumerate}
    \vspace{1cm}
    \begin{outline}
        \1 Время: 10 минут
        \1 Писать и сдавать ручкой на листочке
        \2 Не забудьте подписать
        \1 Максимум 3 балла в плюс к домашкам
    \end{outline}
\end{frame}

\section{Работа с файлами}

\begin{frame}
    \frametitle{Файлы}
    \begin{outline}
        \1 Последовательность байтов на диске
        \2 Бывают \enquote{сырые} и \enquote{текстовые}
        \3 Самому файлу всё равно, это лишь способы интерпретации его содержимого
        \2 Режимы доступа: r, w, a, r+, w+, a+
        \2 Курсор
        \2 EOF
        \1 Функции для работы с файлами:
        \2 fopen, fclose, fprintf, fscanf, fseek, ftell, fgets
        \1 Файлы надо не забывать закрывать
    \end{outline}
\end{frame}

\begin{frame}[fragile]
    \frametitle{Пример, как писать в файл}
    \begin{minted}{cpp}
int main() {
    FILE* out = fopen("ololo.txt", "w");
    if (out == NULL) {
        return -1;
    }
    fwrite("Ololo\n", sizeof(char), 6, out);
    fprintf(out, "%s", "Ololo");
    fclose(out);
    return 0;
}
        \end{minted}
    \begin{outline}
        \1 stdin/stdout --- это тоже файлы
    \end{outline}
\end{frame}

\begin{frame}[fragile]
    \frametitle{Пример, как читать из файла}
    \begin{scriptsize}
        \begin{minted}{cpp}
#include <stdio.h>

int main() {
    FILE *file = fopen("test.txt", "r");
    if (file == NULL) {
        printf("file not found!");
        return 1;
    }
    char *data[100] = {0};
    int linesRead = 0;
    while (!feof(file)) {
        char *buffer = malloc(sizeof(char) * 100);
        const int readBytes = fscanf(file, "%s", buffer);
        if (readBytes < 0) {
            break;
        }
        data[linesRead] = buffer;
        ++linesRead;
    }
    fclose(file);
    ...
}

            \end{minted}
    \end{scriptsize}
\end{frame}

\begin{frame}
    \frametitle{Тонкости}
    \begin{outline}
        \1 Чтение строки целиком: \mintinline{cpp}|fscanf(file, "|\%\mintinline{cpp}|[^\n]", buffer);|
        \1 Или: \mintinline{cpp}|fgets(buffer, sizeof(buffer), file);|
        \1 Working directory
        \2 Обычно Ваша текущая директория в терминале
        \1 Работать с файлом как с массивом элементов некоторого типа может быть удобно
        \2 Можно отобразить файл в память
        \2 В Linux читать man 2 mmap (или info mmap)
    \end{outline}
\end{frame}

\section{Напоминание про модули}

\begin{frame}[fragile]
    \frametitle{Модули}
    \begin{outline}
        \1 Способ группировки кода в логически обособленные группы
        \1 В C это реализуется с помощью заголовочных файлов и файлов с реализацией
        \2 .h и .c
        \1 В отдельный модуль выносятся объявления типов данных и функции, которые делают одно дело
        \2 Например, разные функции сортировки
        \2 Или всё для работы с матрицами
        \1 В интерфейсную часть модуля выносится только то, что может использовать другой код
        \2 Меньше знаешь --- крепче спишь
        \1 Функции, используемые только для реализации, пишутся только в .c-файле
        \2 Например, функция разделения массива для быстрой сортировки или swap
    \end{outline}
\end{frame}

\begin{frame}[fragile]
    \frametitle{Модули}
    \begin{scriptsize}
        Заголовочный файл:
        \begin{minted}{c}
#pragma once

// Комментарий к функции 1
int function1(int x, int y);

// Комментарий к функции 2
void function2();
            \end{minted}
        .c-файл:
        \begin{minted}{c}
#include <имя заголовочного файла.h>

#include <все остальные библиотеки>

int function1(int x, int y)
{
    ...
}

void function2()
{
    ...
}
            \end{minted}
    \end{scriptsize}
\end{frame}

\begin{frame}
    \frametitle{Тонкости}
    \begin{outline}
        \1 Реализации функций в .h-файле писать нельзя
        \2 Иначе будет беда, если один .h-ник подключат в два .c-шника
        \2 Бывают прикольные исключения~--- header-only library
        \1 Комментарии обязательны
        \1 \#pragma once обязательна
        \1 Подключать \enquote{свой} заголовочный файл в .c обязательно
        \1 Файлы .h/.c всегда ходят парами, кроме файла с main
    \end{outline}
\end{frame}

\begin{frame}
    \frametitle{Немного о терминологии}

    \begin{itemize}
        \item Формально в Си нет модулей
        \item Есть~--- единицы трансляции (translation unit)
              \begin{itemize}
                  \item Обычно это .c после препроцессора, именно из них компилятор делает объектные файлы
              \end{itemize}
        \item Единица трансляции определяет область видимости
              \begin{itemize}
                  \item По-умолчанию все функции доступны извне (extern)
                  \item Ключевое слово static у функции делает её невидимой для линковщика, его стоит использовать для внутренних функций
              \end{itemize}
        \item Осторожно: static многозначное ключевое слово
    \end{itemize}

\end{frame}

\section{Системные вызовы}

\begin{frame}
    \frametitle{Системные вызовы}

    \begin{itemize}
        \item Способ обратиться к ОС, обычно для некоторого взаимодействия
        \item Типичные системные вызовы:
              \begin{itemize}
                  \item Выделение памяти
                  \item Запуск приложений
                  \item Работа с файлами (открытие, чтение, запись, закрытие)
                  \item Узнать системную информацию (время, ядро, на котором исполняемся)
              \end{itemize}
        \item Специфичны для ОС
              \begin{itemize}
                  \item Делает приложение непереносимым
                  \item Для Linux читать man 2 syscalls (или info syscalls)
              \end{itemize}
    \end{itemize}

\end{frame}

\section{Виртуальная память}

\begin{frame}
    \frametitle{Идея виртуальной памяти}

    Проблема:
    \begin{itemize}
        \item Памяти мало, процессов много
        \item Хотим удовлетворить всех с минимальными потерями для себя
    \end{itemize}

    Идея:
    \begin{itemize}
        \item Дадим каждому приложению собственное адресное пространство~--- виртуальную память
        \item Отобрать виртуальную память в физическую заставим ОС
    \end{itemize}
\end{frame}

\begin{frame}
    \frametitle{Побочные эффекты}

    \begin{itemize}
        \item Каждое приложение получает свой непрерывный \enquote{массив байтов}, возможно даже большего размера, чем доступно физически
        \item Возможность перемещать данный из сильно ограниченной RAM, например, на SSD
    \end{itemize}

\end{frame}

\begin{frame}
    \frametitle{Возможные определения}
    \framesubtitle{По Я.А. Кириленко}

    Виртуальная организация памяти~--- это процесс расширения логической памяти за пределы физической

    \vspace{1em}

    Виртуальная память~--- это техника, позволяющая исполнять процессы, которые могут находиться в памяти не полностью
\end{frame}

\begin{frame}
    \frametitle{Страничная организация (при поддержке ЦПУ)}
    \framesubtitle{По Я.А. Кириленко}

    Страницы — это части, на которые разбивается пространство виртуальных адресов

    Страницы всегда имеют фиксированный размер (часто 4 КиБ)

    \begin{center}
        \includegraphics[width=0.6\linewidth]{virt_mem.png}
    \end{center}

\end{frame}

\begin{frame}
    \frametitle{Ещё одна красивая картинка}

    Теперь с адресным пространством процесса.
    Полезно, что мы не теряем память между стеком и кучей.%
    \footnote{%
        Картинка из неплохой статьи: \url{https://web.archive.org/web/20161210164858/http://jpauli.github.io/2015/04/16/segmentation-fault.html}\\
        Или на русском: \url{https://habr.com/ru/companies/nix/articles/277759/}}

    \begin{center}
        \includegraphics[width=0.45\linewidth]{virt_mem2.png}
    \end{center}

\end{frame}

\end{document}
