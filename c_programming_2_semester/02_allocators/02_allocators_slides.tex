% !TEX TS-program = lualatex

\documentclass[aspectratio=169, russian]{beamer}

%%% Setup fonts.
\usepackage{fontspec}
%% Fancy fonts
\setmainfont{Iosevka Etoile}
\setsansfont{Iosevka Aile}
\setmonofont{Iosevka}

%% Default fonts
% \setmainfont{CMU Serif}
% \setsansfont{CMU Sans Serif}
% \setmonofont{CMU Typewriter Text}

%% Math
\usepackage{amsmath, amsfonts, amsthm, mathtools} % Advanced math tools.
\usepackage{amssymb}

\usepackage{unicode-math} % Allow TTF and OTF fonts in math and allow direct typing unicode math characters.
\unimathsetup{
    warnings-off={
            mathtools-colon,
            mathtools-overbracket
        }
}
% \setmathfont{Lete Sans Math}[CharacterVariant={3,6},StylisticSet={4}]
\setmathfont{Latin Modern Math}

%%% Language settings.
\usepackage{polyglossia}
\setdefaultlanguage{russian}
\setotherlanguage{english}

%%% Beamer settings
% Themes
\usetheme{Boadilla}
\useinnertheme{circles}
\usecolortheme[style=Latte, accent=Blue]{catppuccin}
% Templates
\setbeamertemplate{navigation symbols}{} %remove navigation symbols
\setbeamertemplate{page number in head/foot}[appendixframenumber]
\setbeamertemplate{title page}[default][colsep=-4bp,rounded=true]

%%% Colors
\hypersetup{colorlinks}
\usepackage{catppuccinpalette}

\usepackage{booktabs}

\usepackage{minted}
\usemintedstyle{catppuccin-latte}

\usepackage{csquotes}

% \usepackage{xurl}

%% Custom commands
\NewDocumentCommand{\attribution}{mmmmm}{\href{#1}{#2}, \href{#3}{#4}, #5}
\NewDocumentCommand{\attributionCCThreeWikimedia}{mm}{\attribution{#1}{#2}{https://creativecommons.org/licenses/by-sa/3.0}{CC BY-SA 3.0}{via Wikimedia Commons}}


\usepackage{xurl}
\usepackage{outlines}

%%% Meta

\title{Аллокаторы}
\author{Николай Пономарев}
\date{14 февраля 2026 г.}
\titlegraphic{\includegraphics[height=1cm]{../фирменный блок_серый.pdf}}
% TODO: Соглашения о вызовах.
\subject{Внутреннее устройство аллокаторов динамической памяти.}

\begin{document}

\begin{frame}[plain, noframenumbering]
    \titlepage
\end{frame}

\section{Аллокаторы}

\begin{frame}{А теперь системное программирование}
    \begin{itemize}
        \item Вы уже имели дело с аллокаторами памяти
        \item Они нужны, чтобы работать со структурами данных динамического размера
        \item Аллокатор работает с кучей (heap)
        \item Для аллокатора куча — множество блоков разного размера
        \item Блок — непрерывный кусок виртуальной памяти, который может быть либо выделенным (allocated), либо свободным (free)
        \item НУО предполагаем, что куча растет по увеличению количества адресов
    \end{itemize}
\end{frame}

\begin{frame}{Какие бывают аллокаторы}
    \begin{block}{Явные (Explicit)}
        Программа явно освобождает выделенный блок \\
        \texttt{malloc} и \texttt{free} в С, \texttt{new} и \texttt{delete} в C++ …
    \end{block}

    \begin{block}{Неявные (Implicit)}
        Аллокатор сам берет на себя обнаружение и освобождение неиспользованных блоков \\
        Сборщики мусора в Java, .NET, Lisp …
    \end{block}

    Мы будем говорить о явных
\end{frame}

\begin{frame}[fragile]{Напомним…}
    \begin{minted}{c}
void* malloc(size_t size)
\end{minted}
    \begin{itemize}
        \item выделяет блок памяти размера не менее size
        \item Return: указатель на выделенный блок или NULL в случае ошибки
        \item Блок памяти обычно выравнивается (в Unix системах~--- по 8 байт)
        \item Не занимается инициализацией! Для этого есть calloc
        \item В рамках лекции считаем размер слова равным 4 байта
    \end{itemize}

    \begin{minted}{c}
void free(*ptr)
\end{minted}
    \begin{itemize}
        \item освобождает блок памяти, на который указывает ptr
        \item ptr должен указывать на начало выделенного блока --- иначе UB
              (если \texttt{NULL}, то ничего не произойдет)
        \item \texttt{free} не сигнализирует об ошибке — с ним надо быть аккуратнее
    \end{itemize}
\end{frame}

\begin{frame}[fragile]{Как аллокатор может выделять память?}
    \begin{itemize}
        \item \texttt{mmap} и \texttt{munmap}
              \begin{itemize}
                  \item Отображение объекта в физической памяти в адресное пространство процесса
                  \item \texttt{/proc/[pid]/maps} — показать отображенные участки памяти процесса
              \end{itemize}
        \item Управление размером кучи: \texttt{void* sbrk(intptr\_t incr)}
              \begin{itemize}
                  \item \texttt{brk} — указатель на конец кучи
                  \item \texttt{sbrk} просто прибавляет \texttt{incr} к этому указателю
                  \item Надо, чтобы запросить у ОС больше памяти в куче
              \end{itemize}
    \end{itemize}
\end{frame}

\begin{frame}{Требования к аллокатору}
    \begin{itemize}
        \item Последовательность запросов \texttt{malloc} и \texttt{free} — произвольная
              \begin{itemize}
                  \item Нельзя полагаться на порядок запросов
                  \item Но мы предполагаем, что \texttt{free} вызывается на участке, который был выделен
              \end{itemize}
        \item Немедленный ответ на запрос
              \begin{itemize}
                  \item Нельзя буферизировать запросы или переупорядочивать
              \end{itemize}
        \item Используется только куча
              \begin{itemize}
                  \item Все нескалярные структуры данных, которыми пользуется аллокатор, должны лежать в куче
              \end{itemize}
        \item Выравнивание блоков
              \begin{itemize}
                  \item Нужно, чтобы в блоке могли размещаться данные любого типа
                  \item В большинстве систем выравнивается по 8 байт
              \end{itemize}
        \item Нельзя модифицировать выделенные блоки
              \begin{itemize}
                  \item Можно манипулировать только свободными блоками
              \end{itemize}
    \end{itemize}
\end{frame}

\begin{frame}{Цели аллокатора}
    \begin{block}{Максимизация пропускной способности (throughput)}
        Количество запросов, выполняющихся в единицу времени \\
        Нужно уменьшать среднее время на запрос к аллокатору
    \end{block}

    \begin{block}{Максимальная утилизация памяти (memory utilization)}
        Полезная нагрузка (payload) — сколько памяти действительно было запрошено \\
        Нам нужно максимизировать суммированную полезную нагрузку для всех запросов относительно размера кучи
    \end{block}

    Эти цели противоречат друг другу. Нужно искать баланс.
\end{frame}

\begin{frame}{Фрагментация}
    Главная причина плохой утилизации кучи \\
    Неиспользованная память не соответствует требованиям запросов аллокатора

    \begin{block}{Внутренняя фрагментация}
        \begin{itemize}
            \item Выделили больше, чем было запрошено (больше, чем payload)
            \item Минимальный размер блока
            \item Выравнивание
        \end{itemize}
    \end{block}

    \begin{block}{Внешняя фрагментация}
        \begin{itemize}
            \item В куче есть место, чтобы выделить память, но нет доступных свободных блоков
            \item Зависит в том числе от будущих запросов
        \end{itemize}
    \end{block}
\end{frame}

\begin{frame}[fragile]{Нюансы реализации}
    Мы могли бы сделать простой аллокатор:
    \begin{itemize}
        \item Куча — массив с указателем p на начало
        \item \texttt{malloc(size)}: увеличить указатель p на size, вернуть новый указатель
        \item \texttt{free(ptr)}: просто return, ничего не делать
    \end{itemize}

    Что мы получили:
    \begin{itemize}
        \item Хорошая пропускная способность, все запросы за константу
        \item Отвратительная утилизация памяти — не переиспользуем свободные блоки
    \end{itemize}
\end{frame}

\begin{frame}{Нюансы реализации}
    Если мы хотим добиться баланса между целями, нужно ответить на следующие вопросы:

    \begin{itemize}
        \item \textbf{Free block organization} — как отслеживать свободные блоки?
        \item \textbf{Placement} — как выбрать свободный блок, куда мы будем аллоцировать?
        \item \textbf{Splitting} — как только заняли свободный блок, что делать с оставшейся частью?
        \item \textbf{Coalescing} — что делать с только что освобожденным блоком?
    \end{itemize}
\end{frame}

\begin{frame}{Implicit Free List}
    Как различать границы блоков и понимать, свободен блок или нет? \\
    Будем хранить все нужное в самом блоке:

    \begin{center}
        \includegraphics[width=0.8\linewidth]{ifl1.png}
    \end{center}
\end{frame}

\begin{frame}
    \frametitle{Как выглядит куча? (Free block organization)}
    \begin{center}
        \includegraphics[width=0.9\linewidth]{ifl2.png}
    \end{center}

    \begin{itemize}
        \item Последовательность занятых и свободных блоков
              \begin{itemize}
                  \item Односвязный список свободных блоков
              \end{itemize}
        \item Время на операцию — линейное от количества \textbf{всех} блоков
              \begin{itemize}
                  \item Это недостаток, с которым в будущем будем бороться
              \end{itemize}
        \item Выравнивание
              \begin{itemize}
                  \item Появляется минимальный размер блока — 2 слова
              \end{itemize}
    \end{itemize}

\end{frame}

\begin{frame}{Размещение выделенного блока (placement) I/II}
    При запросе аллокации — поиск подходящего свободного блока. \\
    А как искать? Есть разные \textbf{политики размещения}!

    \begin{block}{first fit --- Берем первый попавшийся подходящего размера}
        \begin{itemize}
            \item Обычно свободные блоки побольше оказываются в конце списка
            \item В начале свободные блоки меньше => поиск блока побольше займет больше времени
        \end{itemize}
    \end{block}

    \begin{block}{next fit --- Начинаем поиск там, где закончился предыдущий}
        \begin{itemize}
            \item Есть вероятность, что следующий подходящий блок — остаток предыдущего
            \item Может работать быстрее, чем first fit
            \item Хуже утилизирует память
        \end{itemize}
    \end{block}

\end{frame}

\begin{frame}{Размещение выделенного блока (placement) II/II}
    При запросе аллокации — поиск подходящего свободного блока. \\
    А как искать? Есть разные \textbf{политики размещения}!

    \begin{block}{best fit --- Перебираем все блоки и ищем подходящий с наименьшим размером}
        \begin{itemize}
            \item Лучше утилизирует память
            \item Хуже по времени — бежим по всей куче
        \end{itemize}
    \end{block}
\end{frame}

\begin{frame}{Разделение свободных блоков (splitting)}
    Нашли подходящий свободный блок — сколько надо занять?
    \begin{itemize}
        \item Можем весь блок — плохо. Внутренняя фрагментация
        \item Можем делить на 2 части — выделенный блок и остаток
    \end{itemize}

    На примере — запрос на 3 слова:
    \begin{center}
        \includegraphics[width=0.9\linewidth]{splitting.png}
    \end{center}
\end{frame}

\begin{frame}{Получение дополнительной памяти в куче}
    Что, если так и не нашли подходящий свободный блок?

    \begin{enumerate}
        \item Объединение свободных блоков (об этом дальше)
        \item Если не сработало — просим у ОС больше памяти через \texttt{sbrk}
        \item Превращаем новый кусок памяти в большой свободный блок
        \item Вставляем блок в список и используем его для выделения
    \end{enumerate}
\end{frame}

\begin{frame}{Объединение свободных блоков (coalescing)}
    При освобождении свободные блоки могут оказаться рядом:

    \begin{center}
        \includegraphics[width=0.9\linewidth]{coalescing.png}
    \end{center}


    \begin{itemize}
        \item Получили \textbf{ложную фрагментацию}:
              \begin{itemize}
                  \item Запрос на 4 слова не выполнится, хотя место есть
                  \item У нас два блока с payload = 3 слова
              \end{itemize}
        \item Блоки надо объединять
    \end{itemize}
\end{frame}

\begin{frame}{Нюансы объединения}
    Когда это можно делать?

    \begin{block}{Сразу при запросе}
        \begin{itemize}
            \item Быстро
            \item Может привести к лишним действиям
        \end{itemize}
    \end{block}

    \begin{block}{Когда-то позже}
        \begin{itemize}
            \item Если не нашли свободный блок, например
            \item Требует отдельного прохода по куче
        \end{itemize}
    \end{block}
\end{frame}

\begin{frame}{Нюансы объединения}
    Допустим, освободили текущий блок
    \begin{itemize}
        \item Объединиться со следующим — легко за константное время
        \item А как объединяться с предыдущим?
        \item С текущей реализацией — только за линейное время от количества
        \item Получаем довольно медленный free
    \end{itemize}

    Как добиться константы? \\
    \textbf{Нам поможет Дональд Кнут!}
\end{frame}

\begin{frame}{Метод граничных маркеров}
    Улучшим наш блок:

    \begin{center}
        \includegraphics[width=0.65\linewidth]{border_marker1.png}
    \end{center}

    Теперь можем получить размер предыдущего блока!
\end{frame}

\begin{frame}{Метод граничных маркеров --- случаи}
    \begin{center}
        \includegraphics[width=0.7\linewidth]{border_marker2.png}
    \end{center}
\end{frame}

\begin{frame}{Метод граничных маркеров}
    \begin{itemize}
        \item Получили константное время в каждом случае
        \item Подход легко обобщить на разные типы аллокаторов
        \item Тратим много памяти на header и footer
        \item Можем оптимизировать:
              \begin{itemize}
                  \item Можем избавиться от футера у выделенных блоков
              \end{itemize}
    \end{itemize}
\end{frame}

\begin{frame}{Implicit Free Lists --- что получили?}
    \begin{itemize}
        \item Линейное время от количества всех блоков на аллокацию
        \item Константное время на освобождение
        \item Очень просто реализовать
        \item Редко используется из-за скорости malloc, но при этом довольно в определенных случаях может подойти
    \end{itemize}

    Разделение и объединение может распространяться почти на все аллокаторы!
\end{frame}

\begin{frame}{Explicit Free List}
    Строим \textbf{двусвязный список} из свободных блоков!

    \begin{center}
        \includegraphics[width=0.6\linewidth]{efl1.png}
        \includegraphics[width=0.65\linewidth]{efl2.png}
    \end{center}
\end{frame}

\begin{frame}{Explicit Free List — аллокация блока}
    Также, как в Implicit Free List \\
    Опять же, политики размещения бывают разные!
\end{frame}

\begin{frame}{Explicit Free List — освобождение блока}
    Зависит от \textbf{политики вставки в список}

    \begin{block}{Last In First Out (LIFO) --- Вставляем новый блок в начало}
        \begin{itemize}
            \item free работает за константу
        \end{itemize}
    \end{block}

    \begin{block}{Address order --- Блоки в списке упорядочены по адресам}
        \begin{itemize}
            \item free работает за линию — проходим по списку
            \item trade-off: мы лучше утилизируем память: проход first fit приближается к best fit!
        \end{itemize}
    \end{block}

    Граничные маркеры все еще нужны для объединения блоков
\end{frame}

\begin{frame}{Explicit Free List --- что получили?}
    По сравнению с Implicit Free List:
    \begin{itemize}
        \item Улучшили время аллокации: линейное время от количества свободных блоков, а не от количества всех блоков
        \item Нужно больше места на свободный блок, ведь мы храним указатели
        \item Это может увеличивать внутреннюю фрагментацию!
    \end{itemize}

    Время на аллокацию можно еще улучшить!
\end{frame}

\begin{frame}{Segregated Free Lists}
    Массив списков со свободными блоками определенных размеров

    \begin{itemize}
        \item Множество блоков разбиваем на классы по размерам
        \item Политик того, как разбить блоки на классы — много
              \begin{itemize}
                  \item Можно по степеням двойки
                  \item Можно маленькие размеры выделять в отдельные классы
              \end{itemize}
    \end{itemize}

    \begin{center}
        \includegraphics[width=0.6\linewidth]{sfl1.png}
    \end{center}
\end{frame}

\begin{frame}{Segregated Free Lists — аллокация и освобождение}
    \textbf{Аллокация блока размера n:}
    \begin{enumerate}
        \item Ищем в списке свободных блоков подходящего класса
        \item Если нашли свободный блок — выделяем
        \item Разделить, а остаток поместить в нужный список --- опционально
        \item Если не нашли, пробуем искать в списке блоков большего размера
        \item Перебрали все списки и не нашли — что делать?
              \begin{itemize}
                  \item Просим больше памяти у ОС при помощи \texttt{sbrk()}
                  \item Выделяем из новой памяти блок нужного размера
                  \item Остаток выделяем в отдельный блок и вставляем в нужный список
              \end{itemize}
    \end{enumerate}

    \textbf{Освобождение блока:}
    \begin{itemize}
        \item Вставка блока в нужный список
        \item Можем объединять блоки со вставкой в нужный список — опционально
    \end{itemize}
\end{frame}

\begin{frame}{Segregated Free Lists --- что мы уже получили?}
    \begin{itemize}
        \item Время на запрос стало ниже
        \item Теперь исследуем не всю кучу, а какую-то часть
        \item Лучше используем память
        \item Проход с политикой first fit приближает best fit
    \end{itemize}

    То, что мы сейчас описали — концепция. Рассмотрим более конкретные реализации.
\end{frame}

\begin{frame}{Simple Segregated Storage}
    Каждый список хранит блоки одного размера \\
    Пример: размеры \{17-32\} — округляем до 32

    \textbf{Выделение блока:}
    \begin{itemize}
        \item Смотрим нужный список. Если не пуст — берем первый блок. Не разделяем.
        \item Список пуст — запрашиваем у ОС кусок памяти, и делим его на блоки нужного размера, теперь список не пуст.
    \end{itemize}

    \textbf{Освобождение:}
    \begin{itemize}
        \item Вставляем новый свободный блок в начало нужного списка
        \item Никакого объединения
    \end{itemize}
\end{frame}

\begin{frame}{Simple Segregated Storage --- что получили?}
    \begin{itemize}
        \item \texttt{malloc} и \texttt{free} за константное время --- круто!
        \item Уменьшили минимальный размер блока
              \begin{itemize}
                  \item Нам нужен только указатель на следующий блок
              \end{itemize}
        \item Страдаем от внутренней фрагментации
              \begin{itemize}
                  \item Не разделяем же блоки
              \end{itemize}
        \item Страдаем от внешней фрагментации
              \begin{itemize}
                  \item Есть конкретные сценарии
                  \item Много запросов на размер 1, много запросов на размер 2 …
              \end{itemize}
    \end{itemize}
\end{frame}

\begin{frame}{Segregated Fits}
    \begin{itemize}
        \item Каждый список — явный или неявный (как описывалось ранее)
              \begin{itemize}
                  \item В списке — блоки разных размеров!
              \end{itemize}
    \end{itemize}

    \textbf{Выделение блока:}
    \begin{enumerate}
        \item Бежим по нужному списку по политике first fit
        \item Нашли — делим, остаток отправляем в нужный список
        \item Не нашли — ищем в списке класса больших размеров
        \item Перебрали все списки? Просим памяти у ОС, выделяем нужный блок, остаток — помещаем в нужный список
    \end{enumerate}

    \textbf{Освобождение блока:}
    \begin{itemize}
        \item Объединяем блоки и результат отправляем в нужный список
    \end{itemize}
\end{frame}

\begin{frame}{Segregated Fits --- что получили?}
    \begin{itemize}
        \item Поиск не по всей куче, а по ее части
        \item first fit здесь приближается к best fit по всей куче
        \item \textbf{Популярный подход}
              \begin{itemize}
                  \item Используется в пакете malloc стандартной библиотеки Си
              \end{itemize}
    \end{itemize}
\end{frame}

\begin{frame}{Двоичные близнецы}
    \begin{itemize}
        \item Segregated Fits, только каждый класс — степень двойки
              \begin{itemize}
                  \item Округляем размеры
              \end{itemize}
        \item Пусть в куче $2^m$ слов
        \item Держим список для блоков размеров $2^k$, $0 \leq k \leq m$
        \item Изначально у нас один блок размером $2^m$
    \end{itemize}
\end{frame}

\begin{frame}{Двоичные близнецы — выделение блока}
    Допустим, хотим выделить блок размером $2^k$

    \begin{enumerate}
        \item Надо найти первый доступный блок размером $2^j$, $k \leq j \leq m$
        \item Рекурсивно:
              \begin{itemize}
                  \item Если $j = k$ — закончили
                  \item Если нет — рекурсивно делим блок пополам, пока $j$ не станет равным $k$
              \end{itemize}
        \item При делении оставшуюся часть (близнеца) отправляем в нужный список
    \end{enumerate}

    Пример — минимальный размер блока — 64 К, запросили 34 К

    \begin{center}
        \includegraphics[width=1\linewidth]{bin_twins1.png}
    \end{center}
\end{frame}

\begin{frame}{Двоичные близнецы — освобождение блока}
    Объединяем блоки, пока не дойдем до близнеца

    Пример: освобождаем блок D
    \begin{center}
        \includegraphics[width=1\linewidth]{bin_twins2.png}
    \end{center}
\end{frame}

\begin{frame}{Двоичные близнецы — что получили?}
    \begin{itemize}
        \item Быстрый поиск и объединение
        \item Знаем адрес и размер блока — легко посчитать адрес близнеца
              \begin{itemize}
                  \item xxx…x00000 — адрес блока размером 32 байта
                  \item xxx…x10000 — адрес близнеца
                  \item Отличие в одном бите!
              \end{itemize}
        \item Страдаем от внутренней фрагментации
        \item Может подойти, когда размеры выделенных блоков известны и они близки к степеням двойки
    \end{itemize}
\end{frame}

\begin{frame}{Итог}
    Разных аллокаторов много — и они нужны:
    \begin{itemize}
        \item Разным вариантам ОС
        \item СУБД
        \item Реализациям языков программирования
        \item Разным программам на Си
    \end{itemize}

    Здесь рассмотрена только малая часть, их гораздо больше
\end{frame}

\begin{frame}
    \frametitle{Благодарности}

    Рассказ про виртуальную память основан на материалах Я.А. Кириленко

    Рассказ про аллокаторы~--- перевёрстан в \LaTeX{} из презентации В.А. Кутуева

\end{frame}

\end{document}
