% !TEX TS-program = lualatex

\documentclass[aspectratio=169, russian]{beamer}

%%% Setup fonts.
\usepackage{fontspec}
%% Fancy fonts
\setmainfont{Iosevka Etoile}
\setsansfont{Iosevka Aile}
\setmonofont{Iosevka}

%% Default fonts
% \setmainfont{CMU Serif}
% \setsansfont{CMU Sans Serif}
% \setmonofont{CMU Typewriter Text}

%% Math
\usepackage{amsmath, amsfonts, amsthm, mathtools} % Advanced math tools.
\usepackage{amssymb}

\usepackage{unicode-math} % Allow TTF and OTF fonts in math and allow direct typing unicode math characters.
\unimathsetup{
    warnings-off={
            mathtools-colon,
            mathtools-overbracket
        }
}
% \setmathfont{Lete Sans Math}[CharacterVariant={3,6},StylisticSet={4}]
\setmathfont{Latin Modern Math}

%%% Language settings.
\usepackage{polyglossia}
\setdefaultlanguage{russian}
\setotherlanguage{english}

%%% Beamer settings
% Themes
\usetheme{Boadilla}
\useinnertheme{circles}
\usecolortheme[style=Latte, accent=Blue]{catppuccin}
% Templates
\setbeamertemplate{navigation symbols}{} %remove navigation symbols
\setbeamertemplate{page number in head/foot}[appendixframenumber]
\setbeamertemplate{title page}[default][colsep=-4bp,rounded=true]

%%% Colors
\hypersetup{colorlinks}
\usepackage{catppuccinpalette}

\usepackage{booktabs}

\usepackage{minted}
\usemintedstyle{catppuccin-latte}

\usepackage{csquotes}

% \usepackage{xurl}

%% Custom commands
\NewDocumentCommand{\attribution}{mmmmm}{\href{#1}{#2}, \href{#3}{#4}, #5}
\NewDocumentCommand{\attributionCCThreeWikimedia}{mm}{\attribution{#1}{#2}{https://creativecommons.org/licenses/by-sa/3.0}{CC BY-SA 3.0}{via Wikimedia Commons}}


% \usepackage{xurl}

%%% Meta

\title{Битовые операции}
% \author{Николай Пономарев}
% \date{24 ноября 2025 г.}
\date{}
\titlegraphic{\includegraphics[height=1cm]{../фирменный блок_серый.pdf}}
\subject{Биты, байты, переменные, адреса. Битовые операции: конъюнкция, дизъюнкция, отрицание, «исключающее или». Применение битовых операций в задачах: умножение и деление на два, контрольно-восстановительная сумма через операцию «исключающее или» и др. Битовые маски.}

\begin{document}

\begin{frame}[plain, noframenumbering]
    \titlepage
\end{frame}

\begin{frame}
    \frametitle{Типы операций}

    \begin{itemize}
        \item Логические операции: \mintinline{c}{a && b, a || b}, ...
        \item Битовые операции: \mintinline{c}{a & b, a | b}, ...
        \item Таблицы истинности для них одинаковы
    \end{itemize}

\end{frame}

\begin{frame}
    \frametitle{Инверсия}

    \textasciitilde A (логическое \enquote{НЕ}; NOT)

    \vspace{1em}

    \begin{columns}
        \begin{column}{0.45\linewidth}
            \begin{tabular}{@{}lc@{}}
                \toprule
                A & \textasciitilde A \\
                \midrule
                0 & 1                 \\
                1 & 0
            \end{tabular}
        \end{column}
        \begin{column}{0.45\linewidth}
            \begin{tabular}{@{}l}
                100010101110 \\ \midrule
                011101010001
            \end{tabular}
        \end{column}
    \end{columns}

\end{frame}

\begin{frame}
    \frametitle{Конъюнкция}

    A \& B (Логическое \enquote{И}; AND)

    \vspace{1em}

    \begin{columns}
        \begin{column}{0.45\linewidth}
            \begin{tabular}{@{}llc@{}}
                \toprule
                A & B & A \& B \\
                \midrule
                0 & 0 & 0      \\
                0 & 1 & 0      \\
                1 & 0 & 0      \\
                1 & 1 & 1
            \end{tabular}
        \end{column}
        \begin{column}{0.45\linewidth}
            \begin{tabular}{@{}l}
                100010101110 \\
                110101110100 \\ \midrule
                100000100100
            \end{tabular}
        \end{column}
    \end{columns}

\end{frame}

\begin{frame}
    \frametitle{Дизъюнкция}

    A | B (Логическое \enquote{ИЛИ}; OR)

    \vspace{1em}

    \begin{columns}
        \begin{column}{0.45\linewidth}
            \begin{tabular}{@{}llc@{}}
                \toprule
                A & B & A | B \\
                \midrule
                0 & 0 & 0     \\
                0 & 1 & 1     \\
                1 & 0 & 1     \\
                1 & 1 & 1
            \end{tabular}
        \end{column}
        \begin{column}{0.45\linewidth}
            \begin{tabular}{@{}l}
                100010101110 \\
                110101110100 \\ \midrule
                110111111110
            \end{tabular}
        \end{column}
    \end{columns}

\end{frame}

\begin{frame}
    \frametitle{Строгая дизъюнкция}

    A \textasciicircum{} B (Исключающее \enquote{ИЛИ}; XOR)

    \vspace{1em}

    \begin{columns}
        \begin{column}{0.45\linewidth}
            \begin{tabular}{@{}llc@{}}
                \toprule
                A & B & A \textasciicircum{} B \\
                \midrule
                0 & 0 & 0                      \\
                0 & 1 & 1                      \\
                1 & 0 & 1                      \\
                1 & 1 & 0
            \end{tabular}
        \end{column}
        \begin{column}{0.45\linewidth}
            \begin{tabular}{@{}l}
                100010101110 \\
                110101110100 \\ \midrule
                010111011010
            \end{tabular}
        \end{column}
    \end{columns}

\end{frame}

\begin{frame}
    \frametitle{Логический сдвиг влево}

    << n

    \textbf{Пример: } \alert{11}100101 << 2 = 100101\alert{00}

    1 << 8 = ?

\end{frame}

\begin{frame}
    \frametitle{Логический сдвиг вправо}

    >> n

    \textbf{Пример: } 01011\alert{011} >> 3 = \alert{000}01011

\end{frame}

\begin{frame}
    \frametitle{Другие сдвиги}

    \begin{itemize}
        \item Циклический
        \item Арифметический
              \begin{itemize}
                  \item влево: как логический
                  \item вправо: все биты сдвигаются вправо, а на освободившееся место размножается знаковый бит
              \end{itemize}
              \textbf{Пример: } \alert{1}001011\alert{1} >> 1 = \alert{1}1001011
    \end{itemize}

\end{frame}

\begin{frame}
    \frametitle{А что, если...}

    \begin{enumerate}
        \item Сдвинуть на 0 бит? Ничего
        \item Сдвинуть на отрицательное число бит?
        \item Сдвинуть на число битов большее, чем есть в слове?
    \end{enumerate}

\end{frame}

\begin{frame}[plain]

    \begin{center}
        \includegraphics[width=0.9\linewidth]{spec.pdf}
    \end{center}

\end{frame}

\begin{frame}
    \frametitle{Свойства XOR}

    \begin{enumerate}
        \item A \textasciicircum{} A = 0
        \item A \textasciicircum{} 0 = A
    \end{enumerate}

\end{frame}

\begin{frame}
    \frametitle{Помехоустойчивое кодирование}

    X$_1$ \textasciicircum{} X$_2$ \textasciicircum{} X$_3$ = X$_4$

    \vspace{1em}

    Дано: X$_2$, X$_3$, X$_4$

    X$_1$ = ?

\end{frame}

\begin{frame}[fragile]
    \frametitle{Порядок вычислений}

    \begin{minted}{c}
false && func1() && func2()

0 & func1() & func2()

false & func1() & func2()
    \end{minted}

    \begin{enumerate}
        \item Каковы условия корректности выражений?
        \item Будут ли вычисляться значения функций в каждом из случаев?
    \end{enumerate}

\end{frame}

\begin{frame}
    \frametitle{Битовые маски}

    \textbf{Пример:}

    IP адреса: 192.168.1.42/24

\end{frame}

\begin{frame}[fragile]
    \frametitle{Битовые маски}

    \textbf{Пример:}

    Выделение битов контрольной суммы:

    \begin{minted}{c}
bit_i = (a >> i) & 1;
bit_ij = ...;
    \end{minted}

\end{frame}

\begin{frame}
    \frametitle{Задача из практики}

    Эмуляция удаления из списка

    Из [1, 2, 3, 4, 5, 6] хотим получить [1, 3, 5] не удаляя элементы из исходного списка

\end{frame}

\begin{frame}
    \frametitle{Задача на битовые хаки}

    \textbf{Задача:} Проверить, является ли значение переменной x типа \mintinline{c}|long long| степенью двойки.

\end{frame}

\begin{frame}[fragile]
    \frametitle{Вариант 1}

    \begin{minted}{c}
bool check(long long x)
{
    while (x != 1) {
        if (x % 2 == 1)
            return false;
        x = x / 2;
    }
    return true;
}
    \end{minted}

\end{frame}

\begin{frame}[fragile]
    \frametitle{Вариант 2}

    \begin{minted}{c}
bool check(long long x)
{
    double y = log2(x);
    return ((int)y == y);
}
    \end{minted}

\end{frame}

\begin{frame}[fragile]
    \frametitle{Вариант 3}

    \begin{minted}{c}
bool check(long long x)
{
    return x & (x - 1) == 0;
}
    \end{minted}

    \vspace{1em}

    \begin{tabular}{rr}
        x =   & 01001\alert{1}0 \\
        x-1 = & 0100101         \\
        \midrule
        \& =  & 01001\alert{0}0
    \end{tabular}
\end{frame}

\begin{frame}
    \frametitle{Выводы}

    Битовые операции
    \begin{itemize}
        \item Повышают производительность, и иногда значительно
        \item Лежат в основе многих стандартных методов
        \item Внедряются компиляторами при оптимизации
        \item Снижают читаемость кода
    \end{itemize}

\end{frame}

\begin{frame}
    \frametitle{Потенциально рекомендуемая литература}

    Henry Warren \enquote{Hacker's delight}

    \begin{itemize}
        \item Пример про степень двойки взят оттуда
        \item Там же огромное количество других трюков с битами
        \item В книге около 500 страниц, поэтому её слегка страшно рекомендовать в 1 семестре
    \end{itemize}

\end{frame}

\end{document}
