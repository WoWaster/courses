\documentclass[12pt]{scrartcl}

%%% Fonts and language setup.
\usepackage{polyglossia}
% Setup fonts.
\usepackage{fontspec}
\setmainfont{CMU Serif}
\setsansfont{CMU Sans Serif}
\setmonofont{Iosevka}

\usepackage{microtype} % Add fancy-schmancy font tricks

%%% Polyglossia setup after (nearly) everything as described in documentation.
\setdefaultlanguage{russian}
\setotherlanguage{english}

\usepackage{csquotes}

\usepackage{minted}

%%% HyperRef
\usepackage{hyperref}

\title{Диагностический тест по указателям}
\date{}

\begin{document}

% \pagestyle{plain}

\begin{center}
    \usekomafont{title}\huge Диагностическая работа по указателям
\end{center}


\begin{center}
    \itshape

    Результаты данной работы не повлияют на Вашу оценку.

    Проверенные листы будут возвращены Вам.

    Пишите разборчивым почерком.

    Время выполнения~--- 30 минут.

    Желаем удачи!
\end{center}

\begin{enumerate}
    \item Укажите ваши ФИО: \hrulefill
    \item Дайте определение указателя.

          \noindent\makebox[\linewidth]{\rule{\linewidth}{0.4pt}}

          \noindent\makebox[\linewidth]{\rule{\linewidth}{0.4pt}}

          \noindent\makebox[\linewidth]{\rule{\linewidth}{0.4pt}}

    \item Какой тип должен быть у переменной \texttt{y}? \hrulefill
          \begin{minted}{c}
int* x;
y = *x;
        \end{minted}
    \item Зачем нужен нулевой указатель (\mintinline{c}|NULL|)?

          \noindent\makebox[\linewidth]{\rule{\linewidth}{0.4pt}}

          \noindent\makebox[\linewidth]{\rule{\linewidth}{0.4pt}}

          \noindent\makebox[\linewidth]{\rule{\linewidth}{0.4pt}}
    \item Что выведет следующий код? \hrulefill
          \begin{minted}{c}
#include <stdio.h>
int main(void)
{
    int* p, q = 42;
    p = &q;
    (*p)++;
    printf("%d, %d\n", *p, q);
    return 0;
}
        \end{minted}
          %     \item Что выведет следующий код? \hrulefill
          %           \begin{minted}{c}
          % #include <stdio.h>
          % int* f(void)
          % {
          %     int x = 8;
          %     return &x;
          % }
          % int main(void)
          % {
          %     int* y = f();
          %     printf("%d\n", *y);
          %     return 0;
          % }
          %             \end{minted}
    \item Запишите тип указателя на массив указателей на строки без использования нотации \mintinline{c}|[]|. \hrulefill
    \item Дайте определение структуры.

          \noindent\makebox[\linewidth]{\rule{\linewidth}{0.4pt}}

          \noindent\makebox[\linewidth]{\rule{\linewidth}{0.4pt}}

          \noindent\makebox[\linewidth]{\rule{\linewidth}{0.4pt}}
          \newpage
    \item Опишите минимум три способа обратиться к полю \mintinline{c}|name| экземпляра структуры \mintinline{c}|barsik|. \hrulefill

          \noindent\makebox[\linewidth]{\rule{\linewidth}{0.4pt}}

          \begin{minted}{c}
struct Cat {
    char name[33];
    int age;
};
int main(void)
{
    struct Cat barsik;
    struct Cat* barsikPtr = &barsik;

    ...

    return 0;
}
            \end{minted}
    \item Что выведет следующий код? \hrulefill
          \begin{minted}{c}
#include <stdio.h>
struct S {
    int* sf;
};
int* f(struct S* s)
{
    return s->sf;
}
int main(void)
{
    struct S s;
    int* y = f(&s);
    printf("%d\n", *y);
    return 0;
}
\end{minted}
    \item Напишите функцию, которая обменивает значения двух целочисленных переменных. \textit{Сильные духом} могут написать функцию обмена значений двух строковых переменных.
\end{enumerate}


\end{document}
