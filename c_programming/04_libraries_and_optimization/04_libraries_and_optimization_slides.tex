% !TEX TS-program = lualatex

% Основано на презентации Анны Игоревны Васениной и Владимира Александровича Кутуева
% А также презентации Юрия Викторовича Литвинова: https://github.com/yurii-litvinov/courses/tree/master/programming-1st-semester/09-structs-modules-files
\documentclass[aspectratio=169]{beamer}

%%% Setup fonts.
\usepackage{fontspec}
%% Fancy fonts
\setmainfont{Iosevka Etoile}
\setsansfont{Iosevka Aile}
\setmonofont{Iosevka}

%% Default fonts
% \setmainfont{CMU Serif}
% \setsansfont{CMU Sans Serif}
% \setmonofont{CMU Typewriter Text}

%% Math
\usepackage{amsmath, amsfonts, amsthm, mathtools} % Advanced math tools.
\usepackage{amssymb}

\usepackage{unicode-math} % Allow TTF and OTF fonts in math and allow direct typing unicode math characters.
\unimathsetup{
    warnings-off={
            mathtools-colon,
            mathtools-overbracket
        }
}
% \setmathfont{Lete Sans Math}[CharacterVariant={3,6},StylisticSet={4}]
\setmathfont{Latin Modern Math}

%%% Language settings.
\usepackage{polyglossia}
\setdefaultlanguage{russian}
\setotherlanguage{english}

%%% Beamer settings
% Themes
\usetheme{Boadilla}
\useinnertheme{circles}
\usecolortheme[style=Latte, accent=Blue]{catppuccin}
% Templates
\setbeamertemplate{navigation symbols}{} %remove navigation symbols
\setbeamertemplate{page number in head/foot}[appendixframenumber]
\setbeamertemplate{title page}[default][colsep=-4bp,rounded=true]

%%% Colors
\hypersetup{colorlinks}
\usepackage{catppuccinpalette}

\usepackage{booktabs}

\usepackage{minted}
\usemintedstyle{catppuccin-latte}

\usepackage{csquotes}

% \usepackage{xurl}

%% Custom commands
\NewDocumentCommand{\attribution}{mmmmm}{\href{#1}{#2}, \href{#3}{#4}, #5}
\NewDocumentCommand{\attributionCCThreeWikimedia}{mm}{\attribution{#1}{#2}{https://creativecommons.org/licenses/by-sa/3.0}{CC BY-SA 3.0}{via Wikimedia Commons}}


%%% Meta

\title{Библиотеки и оптимизации}
\author{Николай Пономарев}
\date{22 сентября 2025 г.}
\titlegraphic{\includegraphics[height=1cm]{../фирменный блок_серый.pdf}}
\subject{Подключение библиотек. Статическая и динамическая линковка. Среда исполнения языка программирования. Флаги оптимизации -O1, -O2, -O3, -Os.}

\begin{document}

\begin{frame}[plain, noframenumbering]
    \titlepage
\end{frame}

\begin{frame}
    \frametitle{Про сдачу домашек}

    В курсе по Python Вас научили пользоваться Git и GitHub

    \begin{itemize}
        \item Каждая домашняя работа оформляется как Pull Request в свой репозиторий
        \item Добавьте преподавателя в коллабораторы
        \item Одно задание~--- один PR
        \item Следите за оформлением кода, коммитов, комментариев и т.д.
        \item Дедлайн будем считать по дате запроса review (вдруг HwProj опять устанет)
    \end{itemize}

    \begin{alertblock}{Важно}
        Начиная с этой домашки в HwProj должна быть ссылка на Pull Request!
    \end{alertblock}
\end{frame}

\begin{frame}
    \frametitle{Библиотеки}

    \begin{itemize}
        \item Многие задачи решали до Вас неоднократно
        \item Такие готовые решения принято называть \enquote{библиотеками}
        \item Иногда требуется создавать свои библиотеки
        \item Чаще~--- достаточно использовать уже готовые
    \end{itemize}

    \begin{block}{Определение}<2>
        \textit{Библиотекой} в Си называют набор заголовочных файлов, доступных другим приложениям, и файлов с реализацией объявлений из соответствующего заголовочного файла.
        Это могут быть как .c-файлы, так и уже прекомпилированные файлы.
    \end{block}

\end{frame}

\begin{frame}
    \frametitle{Стандартные библиотеки в Си}

    На самом деле Вы использовали уже довольно много разных библиотек:
    \begin{description}
        \item[stdio.h] printf, scanf, ...
        \item[stdlib.h] malloc, calloc, free, ...
        \item[math.h] abs, sin, cos, ...
        \item[string.h] strcmp, strcpy, ...
        \item[stdbool.h] true, false, ...
        \item[] ...
    \end{description}

\end{frame}

\begin{frame}[fragile]
    \frametitle{Связывание со стандартными библиотеками}

    Большая часть функций живёт в libc, однако математика живёт в libm
    \inputminted{c}{workdir/rootOf3.C}
    Нужно собирать как \texttt{gcc -Wall -Wextra -pedantic rootOf3.c -lm -o rootOf3}
\end{frame}

\begin{frame}
    \frametitle{Модули}
    \begin{itemize}
        \item Способ группировки кода в логически обособленные группы
        \item В C это реализуется с помощью заголовочных файлов и файлов с реализацией
              \begin{itemize}
                  \item .h и .c
              \end{itemize}
        \item В отдельный модуль выносятся объявления типов данных и функции, которые делают одно дело
              \begin{itemize}
                  \item Например, разные функции сортировки
                  \item Или всё для работы с матрицами
              \end{itemize}
        \item В интерфейсную часть модуля выносится только то, что может использовать другой код
              \begin{itemize}
                  \item Меньше знаешь --- крепче спишь
              \end{itemize}
        \item Функции, используемые только для реализации, пишутся только в .c-файле
              \begin{itemize}
                  \item Например, функция разделения массива для быстрой сортировки или swap
              \end{itemize}
    \end{itemize}
\end{frame}

\begin{frame}[fragile]
    \frametitle{Модули}

    \begin{columns}[t]
        \begin{column}{0.45\linewidth}
            Заголовочный файл:
            \begin{minted}{c}
#pragma once

// Комментарий к функции 1
int functionOne(int x, int y);

// Комментарий к функции 2
void functionTwo();
            \end{minted}
        \end{column}
        \begin{column}{0.45\linewidth}
            .c-файл:
            \begin{minted}{c}
#include <имя заголовочного файла.h>

#include <все остальные библиотеки>

int functionOne(int x, int y)
{
    ...
}

void functionTwo()
{
    ...
}

            \end{minted}
        \end{column}
    \end{columns}
\end{frame}

\begin{frame}
    \frametitle{Тонкости}
    \begin{itemize}
        \item Реализации функций в .h-файле писать нельзя
              \begin{itemize}
                  \item Иначе будет беда, если один .h-ник подключат в два .c-шника
              \end{itemize}
        \item Комментарии обязательны
        \item \texttt{\#pragma once} обязательна
        \item Подключать \enquote{свой} заголовочный файл в .c обязательно
        \item Файлы .h/.c всегда ходят парами, кроме файла с main
    \end{itemize}
\end{frame}

\begin{frame}[fragile]
    \frametitle{Пример простой собственной библиотеки}

    См. \texttt{workdir/prime}

    \begin{minted}{console}
$ gcc -Wall -Wextra -pedantic main.c -o main
/usr/bin/ld: /tmp/cc4YmGLo.o: in function `main':
main.c:(.text+0x3f): undefined reference to `isPrime'
collect2: error: ld returned 1 exit status
    \end{minted}

    Не выйдет!
    Нужно так:
    \begin{minted}{console}
$ gcc -Wall -Wextra -pedantic prime.c -c
$ gcc -Wall -Wextra -pedantic prime.o main.c -o main
    \end{minted}

\end{frame}

\begin{frame}
    \frametitle{Статическая линковка}

    Способ линковки, при котором библиотеки встраиваются в исполняемый файл

    Плюсы:
    \begin{itemize}
        \item Высокая переносимость
        \item Повышенная безопасность
        \item Скорость исполнения
    \end{itemize}
    Минусы:
    \begin{itemize}
        \item Увеличение размера исполняемого файла
        \item Сложность управления версиями библиотек
    \end{itemize}
\end{frame}

\begin{frame}
    \frametitle{Динамическая линковка}

    Способ связывания, при котором система загружает библиотеки по ходу исполнения программы

    Плюсы:
    \begin{itemize}
        \item Меньший размер исполняемого файла
        \item \enquote{Простота} управления версиями библиотек
    \end{itemize}
    Минусы:
    \begin{itemize}
        \item Проблемы с переносимостью
        \item Временные затраты на поиск и загрузку библиотек по ходу исполнения
    \end{itemize}
\end{frame}

\begin{frame}
    \frametitle{Динамическая линковка vs. Статическая линковка}

    \inputminted[breaklines]{console}{sizes.txt}

\end{frame}

% \begin{frame}
%     \frametitle{\enquote{DLL Hell}}



%     Приложение A использует библиотеку L версии 2.0 и устанавливает её в систему.

%     Приложение B использует библиотеку L

% \end{frame}

\begin{frame}
    \frametitle{Компиляторные оптимизации}

    \texttt{gcc -O1 -O2 -O3 ...}

    \begin{itemize}
        \item раскрытие циклов (избавление от условных переходов)
        \item замена инструкций (умножение на 2 эквивалентно битовому сдвигу)
        \item раскрытие внутренних методов (избавление от вызовов подпрограмм)
        \item использование SSE (одновременная обработка блоков данных)
        \item ...
    \end{itemize}

    \texttt{gcc -Os}~--- оптимизация по размеру

    Compiler Explorer: \url{https://godbolt.org/}

\end{frame}

\begin{frame}
    \frametitle{Домашнее задание}

    Реализовать приложение, сортирующее поступающий в стандартный поток ввода набор целых чисел (не более 100 штук, это гарантируется).
    Числа разделены пробелами, последним символом является перенос строки.

    Процедура сортировки должна быть реализована на языке ассемблера \textit{достаточно оптимальным образом}.
    Кодом возврата приложения является количество элементов, участвовавших в сортировке и изменивших свою позицию.

    Дополнительные соглашения:
    \begin{itemize}
        \item Чтение поступающих на вход сортируемых элементов необходимо производить с помощью команды \texttt{scanf(...)}.
        \item Файлы с кодом приложения должны иметь расширение \texttt{.c}, файл с реализацией сортировки на ассемблере должен иметь расширение \texttt{.s}.
    \end{itemize}

    \alert{Не забудьте приложить инструкцию по сборке!}

\end{frame}

\begin{frame}
    \frametitle{Полезные ссылки}

    \begin{itemize}
        \item C FAQ: \enquote{I'm wondering what to put in .c files and what to put in .h files. (What does ".h" mean, anyway?)}
              \begin{itemize}
                  \item \url{https://c-faq.com/cpp/hfiles.html}
              \end{itemize}
        \item How to handle dynamic and static libraries in Linux
              \begin{itemize}
                  \item \url{https://opensource.com/article/20/6/linux-libraries}
                  \item Статья для тех, кто хочет понять, как создавать библиотеки пригодные для связывания
              \end{itemize}
        \item Про компоновку, dependency hell и обратную совместимость
              \begin{itemize}
                  \item \url{https://habr.com/ru/articles/220961/}
                  \item Про то, к каким побочным эффектам может приводить повсеместное бездумное использование динамической линковки и как их можно лечить
              \end{itemize}
    \end{itemize}

\end{frame}

\end{document}
