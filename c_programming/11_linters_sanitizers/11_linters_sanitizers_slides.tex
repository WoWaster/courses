% !TEX TS-program = lualatex

\documentclass[aspectratio=169, russian]{beamer}

%%% Setup fonts.
\usepackage{fontspec}
%% Fancy fonts
\setmainfont{Iosevka Etoile}
\setsansfont{Iosevka Aile}
\setmonofont{Iosevka}

%% Default fonts
% \setmainfont{CMU Serif}
% \setsansfont{CMU Sans Serif}
% \setmonofont{CMU Typewriter Text}

%% Math
\usepackage{amsmath, amsfonts, amsthm, mathtools} % Advanced math tools.
\usepackage{amssymb}

\usepackage{unicode-math} % Allow TTF and OTF fonts in math and allow direct typing unicode math characters.
\unimathsetup{
    warnings-off={
            mathtools-colon,
            mathtools-overbracket
        }
}
% \setmathfont{Lete Sans Math}[CharacterVariant={3,6},StylisticSet={4}]
\setmathfont{Latin Modern Math}

%%% Language settings.
\usepackage{polyglossia}
\setdefaultlanguage{russian}
\setotherlanguage{english}

%%% Beamer settings
% Themes
\usetheme{Boadilla}
\useinnertheme{circles}
\usecolortheme[style=Latte, accent=Blue]{catppuccin}
% Templates
\setbeamertemplate{navigation symbols}{} %remove navigation symbols
\setbeamertemplate{page number in head/foot}[appendixframenumber]
\setbeamertemplate{title page}[default][colsep=-4bp,rounded=true]

%%% Colors
\hypersetup{colorlinks}
\usepackage{catppuccinpalette}

\usepackage{booktabs}

\usepackage{minted}
\usemintedstyle{catppuccin-latte}

\usepackage{csquotes}

% \usepackage{xurl}

%% Custom commands
\NewDocumentCommand{\attribution}{mmmmm}{\href{#1}{#2}, \href{#3}{#4}, #5}
\NewDocumentCommand{\attributionCCThreeWikimedia}{mm}{\attribution{#1}{#2}{https://creativecommons.org/licenses/by-sa/3.0}{CC BY-SA 3.0}{via Wikimedia Commons}}


% \usepackage{xurl}

%%% Meta

\title{Линтеры и санитайзеры}
\subtitle{Или как стрелять себе в ногу чуть реже}
\author{Николай Пономарев}
\date{10 ноября 2025 г.}
\titlegraphic{\includegraphics[height=1cm]{../фирменный блок_серый.pdf}}
\subject{Статическая и динамическая проверка кода. Линтеры, санитайзеры, фреймворк valgrind и другие инструменты анализа кода.}

\begin{document}

\begin{frame}[plain, noframenumbering]
    \titlepage
\end{frame}

\begin{frame}
    \frametitle{Вспомним уже известное}

    \begin{itemize}
        \item Что такое линтеры?
        \item Зачем они нужны?
        \item А давайте покроем всё тестами!
    \end{itemize}

\end{frame}

\begin{frame}
    \frametitle{Анализ кода}

    \begin{itemize}
        \item<1-> Хотим искать ошибки в коде
        \item<1-> Как это можно делать?
        \item<2-> Тесты
              \begin{itemize}
                  \item Призваны \enquote{ловить} ошибки в логике
                  \item Сложно ловить \enquote{системные} ошибки, вроде работы с памятью
              \end{itemize}
        \item<3-> Статически~--- без запуска кода
              \begin{itemize}
                  \item Статические анализаторы
                  \item Линтер~--- частный случай статического анализатора
                  \item Современные компиляторы подрабатывают статическими анализаторами
              \end{itemize}
        \item<3-> Динамически~--- во время исполнения
              \begin{itemize}
                  \item Динамические анализаторы
              \end{itemize}
    \end{itemize}

\end{frame}

\begin{frame}
    \frametitle{clang-tidy}

    clang-tidy\footnote{Страница: \url{https://clang.llvm.org/extra/clang-tidy/}}~--- линтер для C/C++ на основе экосистемы LLVM
    \begin{itemize}
        \item Требует информации о том, как собирался проект
              \begin{itemize}
                  \item Конфигурируйте CMake c флагом \texttt{-DCMAKE\_EXPORT\_COMPILE\_COMMANDS=ON}
              \end{itemize}
        \item Поддерживает как проверки вида \enquote{предупреждения компилятора} (в частности clang), так и отдельные проверки, например
              \begin{itemize}
                  \item вызовы \mintinline{c}|memset(ptr, 0, n * sizeof(ptr))| вместо \mintinline{c}|memset(ptr, 0, n * sizeof(*ptr))|
                  \item использование магических констант
                  \item правильное использование нейминга
              \end{itemize}
        \item Пример использования (в основном в CI)
              \begin{itemize}
                  \item \url{https://github.com/WoWaster/spbu-c-ci-example}
              \end{itemize}
    \end{itemize}

\end{frame}

\begin{frame}
    \frametitle{Sanitizers}

    \begin{definition}
        Инструментирование~--- модификация (исходного или бинарного) кода программы с целью дальнейшего проведения анализа
    \end{definition}

    \vspace{1em}

    Sanitizers (санитайзеры)\footnote{Репозиторий: \url{https://github.com/google/sanitizers}}~--- набор динамических анализаторов, разработанный Google

    \begin{itemize}
        \item Инструментирование программы на этапе компиляции
        \item Довольно сложно устроены для рассказа сейчас
              \begin{itemize}
                  \item Подробности можно найти в wiki репозитория
                  \item В видео: \url{https://youtu.be/7WyBAUJ8UA8}
              \end{itemize}
    \end{itemize}

\end{frame}

\begin{frame}
    \frametitle{Виды санитайзеров}

    \begin{description}
        \item[ASan] находит ошибки работы с памятью: выход за границы массива, use-after-free, double free
              \begin{itemize}
                  \item В среднем замедляет программу в 2 раза
              \end{itemize}
        \item[MSan] находит использование неинициализированной памяти
              \begin{itemize}
                  \item В среднем замедляет программу в 3 раза
              \end{itemize}
        \item[UBSan] находит ошибки, связанные с undefined behavior
              \begin{itemize}
                  \item Работает на этапе компиляции, поэтому \enquote{бесплатный}
                  \item Есть опциональная библиотека времени исполнения \enquote{немного влияющая на скорость исполнения}
              \end{itemize}
        \item[LSan] находит утечки памяти
              \begin{itemize}
                  \item Отрабатывает перед завершением программы, поэтому почти \enquote{бесплатный}
              \end{itemize}
        \item[другие] ThreadSanitizer, DataFlowSanitizer, TypeSanitizer, RealtimeSanitizer
    \end{description}

\end{frame}

\begin{frame}
    \frametitle{Valgrind}

    Valgrind\footnote{Сайт: \url{https://valgrind.org/}}~--- фреймворк для создания динамических анализаторов, а также набор инструментов

    \begin{itemize}
        \item Не требует пересборки программы, но лучше иметь отладочную информацию
        \item Программа исполняется на \enquote{виртуальном} процессоре, что позволяет собирать огромное количество разной информации о программе
    \end{itemize}

    \vspace{1em}

    Наиболее интересный нам инструмент~--- Memcheck, который позволяет отлавливать ошибки работы с памятью

    Замедляет программу в 10--30 раз

\end{frame}

\begin{frame}
    \frametitle{Практика}

    Разберём примеры отсюда: \url{https://github.com/spbu-coding-2025/workshop-analyzers}

    \vspace{1em}

    Посмотрим на то, как добавить себе clang-format и clang-tidy в репозиторий: \url{https://github.com/WoWaster/spbu-c-ci-example}

\end{frame}

\end{document}
