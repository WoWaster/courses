% !TEX TS-program = lualatex

% Основано на  презентациях Юрия Викторовича Литвинова:
% https://github.com/yurii-litvinov/courses/tree/master/programming-1st-semester/18-console-and-build-systems
\documentclass[aspectratio=169]{beamer}

%%% Setup fonts.
\usepackage{fontspec}
%% Fancy fonts
\setmainfont{Iosevka Etoile}
\setsansfont{Iosevka Aile}
\setmonofont{Iosevka}

%% Default fonts
% \setmainfont{CMU Serif}
% \setsansfont{CMU Sans Serif}
% \setmonofont{CMU Typewriter Text}

%% Math
\usepackage{amsmath, amsfonts, amsthm, mathtools} % Advanced math tools.
\usepackage{amssymb}

\usepackage{unicode-math} % Allow TTF and OTF fonts in math and allow direct typing unicode math characters.
\unimathsetup{
    warnings-off={
            mathtools-colon,
            mathtools-overbracket
        }
}
% \setmathfont{Lete Sans Math}[CharacterVariant={3,6},StylisticSet={4}]
\setmathfont{Latin Modern Math}

%%% Language settings.
\usepackage{polyglossia}
\setdefaultlanguage{russian}
\setotherlanguage{english}

%%% Beamer settings
% Themes
\usetheme{Boadilla}
\useinnertheme{circles}
\usecolortheme[style=Latte, accent=Blue]{catppuccin}
% Templates
\setbeamertemplate{navigation symbols}{} %remove navigation symbols
\setbeamertemplate{page number in head/foot}[appendixframenumber]
\setbeamertemplate{title page}[default][colsep=-4bp,rounded=true]

%%% Colors
\hypersetup{colorlinks}
\usepackage{catppuccinpalette}

\usepackage{booktabs}

\usepackage{minted}
\usemintedstyle{catppuccin-latte}

\usepackage{csquotes}

% \usepackage{xurl}

%% Custom commands
\NewDocumentCommand{\attribution}{mmmmm}{\href{#1}{#2}, \href{#3}{#4}, #5}
\NewDocumentCommand{\attributionCCThreeWikimedia}{mm}{\attribution{#1}{#2}{https://creativecommons.org/licenses/by-sa/3.0}{CC BY-SA 3.0}{via Wikimedia Commons}}


% \usepackage{xurl}

%%% Meta

\title{Системы сборки}
\author{Николай Пономарев}
\date{27 октября 2025 г.}
\titlegraphic{\includegraphics[height=1cm]{../фирменный блок_серый.pdf}}
\subject{Системы сборки. Проблемы ручной сборки программ. Эволюция систем сборки. Make, CMake.}

\begin{document}

\begin{frame}[plain, noframenumbering]
    \titlepage
\end{frame}

\begin{frame}{Системы сборки}
    \begin{itemize}
        \item Среда разработки не всегда доступна
              \begin{itemize}
                  \item Сборочные сервера автоматически выполняют сборку после каждого коммита, там некому открыть IDE и нажать на кнопку \enquote{запустить}
              \end{itemize}
        \item Воспроизводимость сборки
              \begin{itemize}
                  \item Если чтобы собрать программу надо открыть проект, скопировать пару десятков файлов, поправить кое-какие пути и делать это в полнолуние, то возможны ошибки
              \end{itemize}
        \item Автоматизация сборки
              \begin{itemize}
                  \item git clone
                  \item одна консольная команда, которая всё делает за нас
                  \item ...
                  \item готовое к работе приложение
              \end{itemize}
    \end{itemize}
\end{frame}

\begin{frame}[fragile]{Сборка вручную без IDE}

    \begin{minted}{console}
$ gcc main.c
    \end{minted}
    или, например,
    \begin{minted}{console}
$ gcc  -Wall -Wextra -pedantic -O2 -g -c lib1
$ gcc  -Wall -Wextra -pedantic -O2 -g -c lib2
$ gcc  -Wall -Wextra -pedantic -O2 -g -c lib3
$ gcc  -Wall -Wextra -pedantic -O2 -g -lm lib1.o lib2.o lib3.o main.c ...
    \end{minted}

    \vspace{1em}

    \begin{itemize}
        \item Если проект большой, это быстро становится грустно
        \item Десятки тысяч файлов --- не редкость
    \end{itemize}

\end{frame}

\begin{frame}{make}

    \begin{itemize}
        \item Стандарт де-факто по \enquote{низкоуровневым} правилам сборки
        \item Сама ничего не знает про языки программирования, компиляторы и прочие подобные штуки
        \item Знает про цели и зависимости целей
              \begin{itemize}
                  \item Сортирует граф целей
                  \item Цели применяются в порядке от листьев к корню
              \end{itemize}

        \item Правила сборки описываются в Makefile
    \end{itemize}

    {\scriptsize Жаждущие подробностей могут посмотреть на слайды в конце презентации}
\end{frame}

\begin{frame}[fragile]{Пример}
    Синтаксис:
    \begin{minted}{make}
target [target ...]: [component ...]
    [command 1]
    .
    .
    .
    [command n]
    \end{minted}

    \vspace{1em}

    Пример:
    \begin{minted}{make}
hello:
    gcc -Wall -Wextra -pedantic -O2 hello.c -o hello
    \end{minted}

    Запуск:
    \begin{minted}{console}
$ make hello
    \end{minted}
\end{frame}


\begin{frame}{Высокоуровневые системы сборки}
    \begin{itemize}
        \item Либо сами вызывают необходимые инструменты, либо генерируют Makefile
        \item CMake
              \begin{itemize}
                  \item Кроссплатформенная система сборки, очень популярна в С и C++ open source-сообществе
              \end{itemize}
        \item Великое множество других (Bazel, MSBuild, qmake, SCons, Meson, ...)
    \end{itemize}
    Написание скриптов сборки для большого проекта --- отдельная и довольно трудоёмкая задача
\end{frame}

\begin{frame}{CMake}
    \begin{itemize}
        \item Декларативное описание сборки проекта
              \begin{itemize}
                  \item Указываем \textit{что} собирать, а не \textit{как} собирать
              \end{itemize}
        \item С открытым исходным кодом
              \begin{itemize}
                  \item \url{https://gitlab.kitware.com/cmake/cmake}
              \end{itemize}
        \item Прежде всего для сборки C++-проектов, но умеет много чего
        \item Разрабатывается с 1999 года
        \item Сама сборкой не занимается, генерирует конфиг для системы сборки
              \begin{itemize}
                  \item Но может сама их запускать
              \end{itemize}
        \item Пожалуй, основной способ сборки в VS Code (плагином CMake Tools)
              \begin{itemize}
                  \item Ctrl+Shift+P -> CMake: Quick Start
              \end{itemize}
    \end{itemize}
\end{frame}

\begin{frame}{Конфигурация CMake}
    \begin{itemize}
        \item CMakeLists.txt в корне проекта
        \item Также могут быть в подкаталогах (для иерархичной конфигурации)
        \item Бывают также модули (.cmake) и скрипты
        \item Коммитить, соответственно, CMakeLists.txt (все) и .cmake, если есть
    \end{itemize}
\end{frame}

\begin{frame}{Основные понятия}
    \begin{itemize}
        \item Команды
              \begin{itemize}
                  \item \mintinline{cmake}|command_name(список аргументов через пробел)|
                  \item В CMake всё команды
                  \item Всё строки
              \end{itemize}
        \item Переменные
              \begin{itemize}
                  \item \mintinline{cmake}|set(name value)|
                  \item \mintinline{cmake}|message(STATUS "Name = ${name}")|
                  \item Переменные окружения: \mintinline{cmake}|$ENV{Имя}|
              \end{itemize}
        \item Функции
    \end{itemize}
\end{frame}

\begin{frame}{Цели}
    \begin{itemize}
        \item Исполняемые файлы
              \begin{itemize}
                  \item \mintinline{cmake}|add_executable(targetname source1 ...)|
              \end{itemize}
        \item Библиотеки
              \begin{itemize}
                  \item \mintinline{cmake}{add_library(targetname [STATIC | SHARED | ...] source1 ...)}
              \end{itemize}
        \item Линковка целей
              \begin{itemize}
                  \item \mintinline{cmake}{target_link_libraries(myLib [PUBLIC | PRIVATE | ...] dependencyLib)}
              \end{itemize}
    \end{itemize}
\end{frame}

\begin{frame}{Фазы сборки}
    \begin{itemize}
        \item Конфигурация
              \begin{itemize}
                  \item \mintinline{console}|$ cmake <путь до папки с CMakeLists.txt>|
                  \item Читает CMakeLists.txt, строит модель, сохраняет в CMakeCache.txt
                  \item Пытается угадать наличествующие инструменты и их возможности (например, поддерживаемую компилятором версию стандарта)
                  \item Можно вручную подредактировать параметры конфигурации (руками или в cmake-gui)
              \end{itemize}
        \item Генерация --- генерация конфигов для сборки (например, Makefile)
        \item Сборка
              \begin{itemize}
                  \item \mintinline{console}|$ cmake --build|
              \end{itemize}
    \end{itemize}
\end{frame}

\begin{frame}[fragile]{Пример}
    \begin{columns}
        \begin{column}{0.65\linewidth}
            \inputminted[fontsize=\small]{cmake}{cmake_example/CMakeLists.txt}
        \end{column}
        \begin{column}{0.30\linewidth}
            \inputminted{text}{cmake_example/tree.txt}

            \vspace{1em}

            \begin{minted}{console}
$ cmake . -B build
$ cmake --build build
$ ./build/main
            \end{minted}
        \end{column}
    \end{columns}

\end{frame}

\begin{frame}[fragile]{Out-of-source-сборка}
    \begin{columns}
        \begin{column}{0.5\textwidth}
            \begin{minted}{text}
source/
├── build
│   └── сгенерированные файлы
├── CMakeLists.txt
└── исходники
                \end{minted}
        \end{column}
        \begin{column}{0.5\textwidth}
            \begin{minted}{text}
build/
└── сгенерированные файлы
source/
├── CMakeLists.txt
└── исходники
                \end{minted}
        \end{column}
    \end{columns}
\end{frame}

\begin{frame}{Что ещё умеет}
    \begin{itemize}
        \item Ветвления
              \begin{itemize}
                  \item Обычно на основе предопределённых переменных: \url{https://cmake.org/cmake/help/latest/manual/cmake-variables.7.html}
              \end{itemize}
        \item Конфигурации: Debug, Release, RelWithDebInfo, MinSizeRel
              \begin{itemize}
                  \item \mintinline{console}|$ cmake ../MyProject -DCMAKE_BUILD_TYPE=Debug|
              \end{itemize}
        \item Подпапки/подпроекты
              \begin{itemize}
                  \item \mintinline{cmake}|add_subdirectory(sourceDir ...)|
              \end{itemize}
        \item Toolchain-файлы
        \item Скачивать исходники прямо в процессе сборки!
              \begin{itemize}
                  \item FetchContent
                  \item См., например, \url{https://google.github.io/googletest/quickstart-cmake.html}
              \end{itemize}
    \end{itemize}
\end{frame}

\begin{frame}
    \frametitle{Домашнее задание}

    \textbf{Обязательное задание (4 балла):}

    Перенести на CMake домашнее задание к 5 занятию (Продвинутый баланс скобок и Сортировочная станция)

    \vspace{1em}

    \textbf{Дополнительные задания (не входят в общий зачёт):}

    Перенести остальные домашние работы на CMake (1 балл за каждую домашку)

\end{frame}

\appendix

\begin{frame}
    \frametitle{Ещё слайды про make}

    Далее несколько слайдов про make, которые не влезли в пару

\end{frame}

\begin{frame}
    \frametitle{Как работает make}

    make знает про цели, зависимости, временные штампы и правила
    \begin{itemize}
        \item Смотрит на зависимости цели, если у хоть одной временной штамп свежее цели, запускается правило для цели
        \item В процессе цель может обновить свой временной штамп, что приведёт к исполнению правил для зависящих от неё целей
        \item Цели и зависимости образуют направленный ациклический граф (DAG)
        \item make выполняет топологическую сортировку графа зависимостей
        \item Правила применяются в порядке от листьев к корню
    \end{itemize}

\end{frame}

\begin{frame}[fragile]{Продвинутые штуки}
    \begin{itemize}
        \item Переменные
              \begin{itemize}
                  \item \mintinline{make}|MACRO = definition|
                  \item \mintinline{make}|NEW_MACRO = $(MACRO) - $(MACRO2)|
                  \item Переопределение из командной строки
                        \begin{itemize}
                            \item \mintinline{console}|$ make MACRO=ololo|
                        \end{itemize}
              \end{itemize}
        \item Параллельная сборка
              \begin{itemize}
                  \item \mintinline{console}|$ make -j8|
              \end{itemize}
    \end{itemize}
\end{frame}

\begin{frame}[fragile]
    \frametitle{Неявные правила (Implicit rules)}

    \begin{itemize}
        \item Способ описания общих правил
        \item Пример:
              \begin{minted}{make}
%.txt : %.docx
    pandoc $< -o $@
                \end{minted}
              Использование: \mintinline{console}|$ make test.txt|, при существовании \texttt{test.docx}
        \item Реализованы для множества языков (C, C++, Pascal, Fortran, ...)
        \item Можно собрать однофайловую программу на Си: \mintinline{console}|$ make main|
        \item Или даже собрать небольшой проект из нескольких файлов:
              \begin{minted}{make}
CFLAGS += -Wall -Wextra -pedantic -O2 -g
main : lib1.o lib2.o lib3.o
        \end{minted}
    \end{itemize}

    Подробнее: \url{https://www.gnu.org/software/make/manual/make.html#Implicit-Rules}

\end{frame}


\end{document}
