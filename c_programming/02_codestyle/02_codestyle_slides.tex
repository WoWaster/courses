% !TEX TS-program = lualatex

% Основано на презентации Юрия Викторовича Литвинова
% https://github.com/yurii-litvinov/courses/tree/master/programming-1st-semester/03-styleguide
\documentclass[aspectratio=169]{beamer}

%%% Setup fonts.
\usepackage{fontspec}
%% Fancy fonts
\setmainfont{Iosevka Etoile}
\setsansfont{Iosevka Aile}
\setmonofont{Iosevka}

%% Default fonts
% \setmainfont{CMU Serif}
% \setsansfont{CMU Sans Serif}
% \setmonofont{CMU Typewriter Text}

%% Math
\usepackage{amsmath, amsfonts, amsthm, mathtools} % Advanced math tools.
\usepackage{amssymb}

\usepackage{unicode-math} % Allow TTF and OTF fonts in math and allow direct typing unicode math characters.
\unimathsetup{
    warnings-off={
            mathtools-colon,
            mathtools-overbracket
        }
}
% \setmathfont{Lete Sans Math}[CharacterVariant={3,6},StylisticSet={4}]
\setmathfont{Latin Modern Math}

%%% Language settings.
\usepackage{polyglossia}
\setdefaultlanguage{russian}
\setotherlanguage{english}

%%% Beamer settings
% Themes
\usetheme{Boadilla}
\useinnertheme{circles}
\usecolortheme[style=Latte, accent=Blue]{catppuccin}
% Templates
\setbeamertemplate{navigation symbols}{} %remove navigation symbols
\setbeamertemplate{page number in head/foot}[appendixframenumber]
\setbeamertemplate{title page}[default][colsep=-4bp,rounded=true]

%%% Colors
\hypersetup{colorlinks}
\usepackage{catppuccinpalette}

\usepackage{booktabs}

\usepackage{minted}
\usemintedstyle{catppuccin-latte}

\usepackage{csquotes}

% \usepackage{xurl}

%% Custom commands
\NewDocumentCommand{\attribution}{mmmmm}{\href{#1}{#2}, \href{#3}{#4}, #5}
\NewDocumentCommand{\attributionCCThreeWikimedia}{mm}{\attribution{#1}{#2}{https://creativecommons.org/licenses/by-sa/3.0}{CC BY-SA 3.0}{via Wikimedia Commons}}


%%% Meta

\title{Стайлгайд}
\author{Николай Пономарев}
\date{11 сентября 2025 г.}
\titlegraphic{\includegraphics[height=1cm]{../фирменный блок_серый.pdf}}
\subject{Стиль кодирования. Процесс сборки и запуска программы — компилятор, линковщик, IDE. Практика, написание первых программ на C.}

\begin{document}

\begin{frame}[plain, noframenumbering]
    \titlepage
\end{frame}

\begin{frame}
    \frametitle{Что такое красивый код?}

    \begin{exampleblock}{Общее правило хорошего кода}
        Программы пишутся для людей, а не для компьютера
    \end{exampleblock}

    \vspace{1em}

    \begin{itemize}
        \item Правила хорошего кода
              \begin{itemize}
                  \item Best practices: общие и специфические для языка
              \end{itemize}
        \item Консистентное форматирование в соответствии с инструкцией
              \begin{itemize}
                  \item Использование code style и форматеров
              \end{itemize}
    \end{itemize}

\end{frame}

\begin{frame}
    \frametitle{Правила хорошего кода}

    \begin{itemize}
        \item \enquote{Школьник-стайл} именования переменных (a, b, c1)
        \item Компилироваться без предупреждений
              \begin{itemize}
                  \item используйте флаги компилятора: \texttt{-Wall -Wextra -pedantic}
              \end{itemize}
        \item Не должно быть копипаста
        \item Одна сущность должна играть одну роль в программе
              \begin{itemize}
                  \item Одна функция должна делать одно дело
                  \item Одна переменная должна означать что-то одно
              \end{itemize}
        \item По возможности сужайте области видимости переменных
        \item Используйте самые узкие типы из подходящих
        \item Переменные --- это плохо, константы --- хорошо
        \item Глобальные переменные --- это очень, очень плохо
        \item goto --- это вообще ужасно
    \end{itemize}

\end{frame}

\begin{frame}
    \frametitle{Форматирование кода}

    \begin{itemize}
        \item Отступы!
        \item Пробелы!
        \item Правила именования
              \begin{itemize}
                  \item Переменные со строчной, типы с заглавной
                  \item camelCase
              \end{itemize}
        \item Один оператор на одной строке%, и побольше фигурных скобок
        \item Бинарные операторы и ключевые слова выделяются пробелами
              % \item Стайлгайд выложен на HwProj, прочитайте и следуйте
    \end{itemize}

    \begin{exampleblock}{Стиль кода для Си}<2>
        В курсе будем использовать стиль WebKit
    \end{exampleblock}
\end{frame}

\begin{frame}
    \frametitle{Один из друзей программиста~--- форматер}

    \begin{itemize}
        \item Многие вещи из code style~--- дело техники
        \item Современный программист пользуется большим количеством инструментов
        \item Форматер один из из них
        \item Для Си будем использовать ClangFormat
    \end{itemize}

\end{frame}


\begin{frame}
    \frametitle{Домашнее задание}

    \begin{enumerate}
        \item Написать программу, считающую значение формулы $x^4 + x^3 + x^2 + x + 1$ за два умножения
        \item Написать программу нахождения неполного частного от деления $a$ на $b$ (целые числа), используя только операции сложения, вычитания и умножения
        \item Дан массив целых чисел $x[1] \dots x[m + n]$, рассматриваемый как соединение двух его отрезков: начала $x[1] \dots x[m]$ длины $m$ и конца $x[m + 1] \dots x[m + n]$ длины $n$.
              Не используя дополнительных массивов, переставить начало и конец %(обращением двух частей массива, а потом его самого)
        \item Посчитать число \enquote{счастливых билетов} (билет считается \enquote{счастливым}, если сумма первых трёх цифр его номера равна сумме трёх последних), подсчётом числа билетов с заданной суммой трёх цифр
    \end{enumerate}

    \alert{Ваш код должен соответствовать стилю WebKit!}
\end{frame}

\begin{frame}
    \frametitle{Полезные ссылки}

    \begin{itemize}
        \item WebKit Code Style Guidelines
              \begin{itemize}
                  \item \url{https://webkit.org/code-style-guidelines/}
                  \item Прочитайте целиком, но фильтруйте информации о C++ и Objective-C
              \end{itemize}
        \item ClangFormat~--- форматер для Си
              \begin{itemize}
                  \item \url{https://clang.llvm.org/docs/ClangFormat.html}
                  \item Настройте его в своей любимой среде
                  \item В Visual Studio: \url{https://learn.microsoft.com/en-us/visualstudio/ide/reference/options-text-editor-c-cpp-formatting?view=vs-2022}
              \end{itemize}
        \item Правила хорошего кода от Юрия Викторовича Литвинова
              \begin{itemize}
                  \item \url{https://github.com/yurii-litvinov/courses/blob/master/programming-1st-semester/styleguide.md}
                  \item Местами не сответствует стилю WebKit, игнорируйте
              \end{itemize}
    \end{itemize}

\end{frame}

\end{document}
