% !TEX TS-program = lualatex

% Основано на презентации Юрия Викторовича Литвинова
% https://github.com/yurii-litvinov/courses/tree/master/programming-1st-semester/02-intro
\documentclass[aspectratio=169]{beamer}

%%% Setup fonts.
\usepackage{fontspec}
%% Fancy fonts
\setmainfont{Iosevka Etoile}
\setsansfont{Iosevka Aile}
\setmonofont{Iosevka}

%% Default fonts
% \setmainfont{CMU Serif}
% \setsansfont{CMU Sans Serif}
% \setmonofont{CMU Typewriter Text}

%% Math
\usepackage{amsmath, amsfonts, amsthm, mathtools} % Advanced math tools.
\usepackage{amssymb}

\usepackage{unicode-math} % Allow TTF and OTF fonts in math and allow direct typing unicode math characters.
\unimathsetup{
    warnings-off={
            mathtools-colon,
            mathtools-overbracket
        }
}
% \setmathfont{Lete Sans Math}[CharacterVariant={3,6},StylisticSet={4}]
\setmathfont{Latin Modern Math}

%%% Language settings.
\usepackage{polyglossia}
\setdefaultlanguage{russian}
\setotherlanguage{english}

%%% Beamer settings
% Themes
\usetheme{Boadilla}
\useinnertheme{circles}
\usecolortheme[style=Latte, accent=Blue]{catppuccin}
% Templates
\setbeamertemplate{navigation symbols}{} %remove navigation symbols
\setbeamertemplate{page number in head/foot}[appendixframenumber]
\setbeamertemplate{title page}[default][colsep=-4bp,rounded=true]

%%% Colors
\hypersetup{colorlinks}
\usepackage{catppuccinpalette}

\usepackage{booktabs}

\usepackage{minted}
\usemintedstyle{catppuccin-latte}

\usepackage{csquotes}

% \usepackage{xurl}

%% Custom commands
\NewDocumentCommand{\attribution}{mmmmm}{\href{#1}{#2}, \href{#3}{#4}, #5}
\NewDocumentCommand{\attributionCCThreeWikimedia}{mm}{\attribution{#1}{#2}{https://creativecommons.org/licenses/by-sa/3.0}{CC BY-SA 3.0}{via Wikimedia Commons}}


%%% Meta

\title{Введение}
\subtitle{Си, разбор задач}
\author{Николай Пономарев \and Юрий Литвинов}
\date{4 сентября 2025 г.}
\titlegraphic{\includegraphics[height=1cm]{../фирменный блок_серый.pdf}}
\subject{Введение в C, структура программы, основные языковые конструкции (функции и рекурсия, переменные, элементарные типы и арифметические операции, массивы, указатели, строки, ввод-вывод с консоли), среда разработки.}

\begin{document}

\begin{frame}[plain, noframenumbering]
    \titlepage
\end{frame}

\begin{frame}
    \frametitle{Формальные вопросы}
    \begin{columns}
        \begin{column}{0.65\linewidth}
            \begin{itemize}
                \item Занятия по четвергам в 3389 у обеих подгрупп
                \item Берите с собой ноутбуки
                \item Курс на HwProj: \url{https://hwproj.ru/courses/50056}
                      \begin{itemize}
                          \item Там надо зарегистрироваться и подать заявку на курс
                          \item Используйте человеческие имя и фамилию, желательно по-русски
                      \end{itemize}
                \item Условия домашек и материалы с пар будут там, сдавать задачи туда же

            \end{itemize}
        \end{column}
        \begin{column}{0.3\linewidth}
            \includegraphics[width=0.95\linewidth]{hwproj.pdf}
        \end{column}
    \end{columns}

\end{frame}

\begin{frame}
    \frametitle{Контакты}

    \begin{columns}[t]
        \begin{column}{0.45\linewidth}
            Пономарев Николай Алексеевич
            \begin{itemize}
                \item Почта: \href{mailto:n.ponomarev@spbu.ru}{n.ponomarev@spbu.ru}
                \item Telegram: \href{https://t.me/wowaster}{@wowaster}
                \item Комната: 3250 (кубик 3248)
                      \begin{itemize}
                          \item Пишите заранее!
                          \item К октябрю может появиться выделенный таймслот
                      \end{itemize}
            \end{itemize}
        \end{column}
        \begin{column}{0.45\linewidth}
            Юрий Викторович Литвинов
            \begin{itemize}
                \item Почта: \href{mailto:y.litvinov@spbu.ru}{y.litvinov@spbu.ru}
                \item Telegram: \href{https://t.me/yurii\_litvinov}{@yurii\_litvinov}
            \end{itemize}
        \end{column}
    \end{columns}

    \begin{center}
        \Large
        \alert{Пишите по любому вопросу!}
    \end{center}

    Если используете не своё имя в Telegram, представляйтесь, кто Вы и откуда!
\end{frame}

\begin{frame}
    \frametitle{Критерии оценивания}
    \begin{itemize}
        \item Шкала оценивания ECTS, оценки от A до F
        \item Надо набирать баллы:
              \begin{itemize}
                  \item За домашки (их будет много!)
                  \item За две контрольные
                  \item За зачёт, который по сути большая контрольная
              \end{itemize}
        \item Итоговый балл за домашки: $\max(0, (n/N\ –\ 0.6)) \times 2.5 \times 100$
              \begin{itemize}
                  \item Если сделано меньше 60\%~--- это 0, если 80\%~--- 50 баллов
                  \item Зачёт~--- строго больше 50 баллов, так что 80\%~--- минимум
              \end{itemize}
        \item Есть дедлайны (минус балл к максимуму за каждую неделю, но не больше половины баллов)
        \item Итоговый балл за контрольные: $n/N \times 100$, их можно переписывать
        \item Балл за зачёт считается так же, но переписывать можно только трижды
        \item В качестве итогового берётся \textbf{минимум} из этих баллов
    \end{itemize}
\end{frame}

\begin{frame}
    \frametitle{Шкала оценивания ECTS}

    \begin{center}
        \begin{tabular}{rl}
            \toprule
            Балл    & Оценка ECTS  \\
            \midrule
            90--100 & A            \\
            80--89  & B            \\
            70--79  & C            \\
            61--69  & D            \\
            50--60  & E            \\
            0--50   & на пересдачу \\
            \bottomrule
        \end{tabular}
    \end{center}

\end{frame}

\begin{frame}
    \frametitle{Что будет в I семестре}
    \begin{itemize}
        \item Ликвидация безграмотности по программированию на Си
        \item Отладка и тестирование
        \item Инструменты разработчика
        \item Внутреннее представление данных
        \item Работа с указателями, стеки, очереди, списки и т.п.
    \end{itemize}
\end{frame}

\begin{frame}
    \frametitle{Советы по организации работы}

    \begin{itemize}
        \item Windows хорош наличием Visual Studio со встроенным компилятором
        \item Установите себе Linux
              \begin{itemize}
                  \item EndeavourOS: \url{https://endeavouros.com} (помни про \url{https://wiki.archlinux.org})
                  \item Ubuntu: \url{https://ubuntu.com} (Debian тоже хорош!)
                  \item WSL2 (не совсем Linux)
              \end{itemize}
        \item Под Linux вполне хватит компилятора gcc или clang и Visual Studio Code с расширением для Си (Microsoft или clangd)
        \item Используйте командную оболочку
              \begin{itemize}
                  \item Linux: bash, zsh, fish
                  \item Windows: cmd.exe, PowerShell, MSYS2
              \end{itemize}
        \item И утилиты (Linux-only): ls/zoxide, nano/vim, cat/bat, less, grep/ripgrep, find/fd, mc/yazi
    \end{itemize}

\end{frame}

\begin{frame}
    \frametitle{Небольшое введение в Си}

    \inputminted{c}{01_basic_c.c}

\end{frame}

\begin{frame}
    \frametitle{Как запустить}

    \inputminted{console}{01_basic_c_trace.txt}

\end{frame}

\begin{frame}
    \frametitle{Указатели}

    \inputminted{c}{01_pointers_question.c}

    \vspace{1em}

    \alert{Что выведет данная программа?}

\end{frame}

\begin{frame}
    \frametitle{Указатели для передачи параметров}

    \inputminted[fontsize=\small]{c}{01_pointers_basic.c}

\end{frame}

\begin{frame}
    \frametitle{Домашнее задание}

    \begin{block}{Задача № 1}
        Написать \enquote{Hello, world!} на Си
    \end{block}

    Примерный план решения:
    \begin{enumerate}
        \item Зарегистрироваться на HwProj
        \item Установить компилятор
        \item Настроить среду разработки
        \item Написать код
        \item Скомпилировать программу
        \item Запустить программу
        \item Сделать скриншот работы
        \item Сдать на HwProj текст программы и скриншот
    \end{enumerate}

\end{frame}

\begin{frame}
    \frametitle{Полезные ссылки}

    \begin{itemize}
        \item Спецификация Си с примерами
              \begin{itemize}
                  \item \url{https://en.cppreference.com/w/c.html}
              \end{itemize}
        \item ВикиКнига \enquote{C Programming}
              \begin{itemize}
                  \item \url{https://en.wikibooks.org/wiki/C_Programming}
                  \item Раздел \enquote{Intro exercise}~--- подробная инструкция по выполнению домашнего задания
              \end{itemize}
        \item Туториал по Си от команды среды рабочего стола для Linux Enlightenment
              \begin{itemize}
                  \item \url{https://www.enlightenment.org/docs/c/start}
              \end{itemize}
    \end{itemize}
\end{frame}

\appendix
\begin{frame}
    \frametitle{Условия задач с теста (если останется время)}
    \begin{enumerate}
        \item Написать алгоритм нахождения неполного частного от деления $a$ на $b$ (целые числа), используя только операции сложения, вычитания и умножения.
        \item Подсчитать число \enquote{счастливых билетов} (билет считается \enquote{счастливым}, если сумма первых трёх цифр его номера равна сумме трёх последних).
        \item Написать алгоритм проверки баланса скобок в исходной строке (т.е. число открывающих скобок равно числу закрывающих и выполняется правило вложенности скобок).
        \item Какое наименьшее количество операции умножения достаточно для вычисления значения формулы $x^4 + x^3 + x^2 + x + 1$?
    \end{enumerate}
\end{frame}

\end{document}
